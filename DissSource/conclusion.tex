\section{Conclusion}

The aim of this project was to implement various constructive colouring algorithms that provide solutions to the GCP. It also aimed to test these algorithms upon multiple test instances to gain a deeper understanding into their efficiency and efficacy. It was anticipated that the findings of this project would provide insights into which constructive algorithm may be the most appropriate choice in a given context. The results show that, if efficiency is not a concern RLF provides the most efficacious colourings in most contexts. However, for very small and sparse graphs, DSATUR was the most efficacious. If efficiency is a concern, then GREEDY proves to be the most efficient. However, the solutions it provides are far from optimal in comparison to other algorithms. The WELSH-POWELL algorithm proves to be similarly efficient to the GREEDY algorithm, but produces slightly better colourings (i.e. fewer colours used). However, perhaps the best all-rounder when considering both efficiency and efficacy for all contexts is the DSATUR algorithm. 
\\\\
This project also aimed to produce a piece of software that would allow a user to run similar tests i.e. to run the implemented algorithms on various test instances. The development of this software was described in section 3. This software runs as intended and fulfils the objectives and aims set out by this project. Due to the time constraints of this project, most time was allocated to ensuring that the algorithms were implemented correctly and that the software was working as intended, with less time allocated to fully consider user experience. Therefore, the system (whilst being fully functional) may be improved by future consideration of how a user may interact with it. Potential improvements may include implementing a GUI for the user to interact with, and addition of visual animations of the algorithms finding solutions in real time. 
\\\\
A set of objectives were laid out in section 1.2 of this project. The first objective was to conduct a literature review to better understand the GCP and the various techniques employed in solving it. This objective was met, with a thorough review of the literature being conducted in section 2. This review became the basis for the choice of algorithms, design of the system, system implementation, and guided the metrics used to measure algorithm performance. Objectives two, three and four were to implement four constructive graph colouring algorithms, implement a method of reading in graph instances for algorithms to run on, and implement a way to generate random graphs. These objectives were met by the construction of the software, and this process is shown in section 3. Objective four was to thoroughly test the software. This objective was met in part i.e. whilst it has been established that the software worked as expected and results generated were valid, time constraints made thorough testing unfeasible. This is an area that could be improved upon in future research. Finally, objective five was to visualise and discuss the results generated from running the algorithms on multiple random graph instances. This objective was also met, and discussed in detail in section 4. 

\subsection{Retrospective Review}
One of the main things learned during this project is how to plan, manage, conduct, and report on a large piece of research in the field of computer science. Whilst the author has conducted several smaller projects in this field, this project has allowed for a greater understanding of the processes of conducting large pieces of research. In addition, this project has allowed the author to gain a deeper understanding of Graph Theory, in particular the GCP. This has sparked a great interest in learning more about optimisation and decision-based problems. This project has also allowed the author to build upon their knowledge of designing and implementing algorithms using various data structures, and modifying an algorithms heuristics to generate more desirable outputs. 
\\\\
A challenge of this project came during the implementation phase. The problem of colouring graphs seems a simple task for a human i.e. a human could colour a graph manually and understand the heuristics of an algorithm, but translating this into code proves more challenging. It required a substantial amount of research and development of personal knowledge to understand the specific techniques and data structures required to be be able to implement these heuristics in a manner that a computer could understand. Whilst this was challenging, it also proved to be immensely rewarding to understand and develop skills in this fundamental aspect of computing science.


\subsection{Follow on work}
Potential areas for follow on work have been mentioned previously. This includes improvement of the user interface to better consider how a user may interact with it. Suggestions for improvements include implementing a GUI for the user to interact with, and addition of visual animations of the algorithms finding solutions in real time. Further research can also be conducted into the GCP and implementing algorithms that use meta-heuristics in finding solutions to the GCP, such as those suggested in section 5.2. Perhaps in the future work could also be conducted on designing a new heuristic for tackling the GCP. 